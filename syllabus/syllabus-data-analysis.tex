\documentclass{article}
    \usepackage[margin=0.7in]{geometry}
    \usepackage[parfill]{parskip}
    \usepackage[utf8]{inputenc}
    \usepackage{amsmath,amssymb,amsfonts,amsthm, fontawesome5, natbib}
\renewcommand\refname{Indicative Reading}
\begin{document}

\begin{center}
    {\Large\textbf{3ACMT-DA | Data Analysis} (\textit{September 2021})}

This syllabus is tentative and subject to change.

{\small
    \faClock~Class Meetings: A.116, Friday, 8:00--9:30am CET; Office Hours: B.140, Tuesday, 2--5pm CET;
}

\end{center}

\begin{itemize}
    \item \faUser~Mickael Temporão, Associate Professor
    \item \faEnvelope~\textit{m.temporao@sciencespobordeaux.fr}
\end{itemize}


\subsection*{Description}
Do social media polarize the electorate? How can we measure ideology? How can we predict election outcomes? What factors explain voting behaviour? These are just a few of the numerous questions that social scientists are tackling with quantitative data. Beyond academia, companies and non-profits have invested heavily in data science techniques to learn about their users, platforms, and programs. Data scientists at these institutions are essentially applied social scientists and employ many of the same techniques you will learn in this course.

What will you learn in this course?
Our goal is to give you the ability to understand, explain, and perform modern social science research with a special focus on data analysis and inference. You will be able to read and understand the methodology of most academic articles in the social sciences, but more importantly you will have a foot in the door of the data science world. The ability to collect and analyze data in a sophisticated manner is becoming a crucial skill set for the modern job market across industries. Finally, you will obtain data literacy that will help you be a critical consumer of evidence for the rest of your life.

Learning support will be provided for at least one programming language, such as \textit{C++}, \textit{Julia}, \textit{Python}, or \textit{R}, but the choice of language supported may vary between years, depending on judged benefits to students, whether in terms of pedagogy or resulting skills. This year, the default choice is \textit{Python}.

\subsection*{Objectives}

% In this course, you will learn to:

\begin{itemize}
    \item Visualize, summarise, and analyse real-world data using reproducible code based on cutting-edge open source tools for data analysis.
    \item Empirically test theories, including the derivation of hypotheses, conceptualization, measurement and inference.
    \item Understand the scientific method and critically evaluate scientific information.
\end{itemize}

\subsection*{Prerequisites}
The most important prerequisites is motivation to work hard on likely unfamiliar material. The course is designed for social science students with no previous experience of quantitative methods, statistics or computer programming. By the end of the course students should be able to carry out univariate and bivariate data analysis and have an appreciation of multiple linear regression and statistical inference.

\subsection*{Assessment}

The course follows a “learn-by-doing” approach and places emphasis on gaining experience by solving real-world problems by analyzing real-world data and applying them to your own research project.

The grades for this course are obtained through continuous assessment, a series of activities that must be completed throughout the year. The three main activities are participation (30\%), coding challenges, (30\%) and research project (40\%). Each of these activities are broken down into several smaller incremental steps to help you make progress and succeed.

\bibliographystyle{apsr}
\bibliography{/home/mt/references.bib}
\nocite{king1994designing, imai2018quantitative,gelman2020regression}

\end{document}

